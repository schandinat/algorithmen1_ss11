\documentclass{beamer}
\usepackage{ngerman}
\usepackage{graphicx}
\usepackage{listings}
\usepackage{amsmath}
\usepackage{amssymb}
\usetheme{Madrid}
\title{Algorithmen I Tutorium}
\author{Florian Tobias Schandinat}
\date{19.05.2011}
\institute{FTS}
\lstset{basicstyle=\small\ttfamily,tabsize=4,showstringspaces=false}


\begin{document}


\begin{frame}
\frametitle{Willkommen}
\begin{block}{Algorithmen I Tutorium 19}
\begin{description}
\item[Wer?] Florian Tobias Schandinat\\
\item[Wo?] 50.34, Raum -118\\
\item[Wann?] jeden Donnerstag 15:45-17:15
\end{description}
\end{block}

\begin{block}{Material online}
http://github.com/schandinat/algorithmen1\_ss11
\end{block}
\end{frame}

\begin{frame}
\frametitle{B"aume}
\begin{block}{Grundlagen}
\begin{itemize}
\item Wurzel, Bl"atter
\item azyklisch
\item Weg von Wurzel zu jedem Knoten eindeutig
\end{itemize}
\end{block}

\pause

\begin{exampleblock}{Vielf"altig einsetzbar}
\begin{itemize}
\item 5 + 4 * 3 - 2 * (1 - 2)\pause
\item \{42, 23, 1, 2, 3, 99\}\pause
\item \{Paul, Peter, Luise, Anna\}\pause
\item ...
\end{itemize}
\end{exampleblock}

\pause

\begin{block}{Suchb"aume}
Knoten erf"ullen eine Ordnungsrelation
\end{block}
\end{frame}


\begin{frame}
\frametitle{Bin"are B"aume}
\begin{alertblock}{Einschr"ankung}
Maximal 2 Kinder/Knoten
\end{alertblock}

\pause

\begin{block}{Darstellung im Array}
Beziehungen von A[i] \textbf{(sofern vorhanden)}
\pause
\begin{description}
\item[Elternknoten] $A[\lfloor\frac{i}{2}\rfloor]$
\item[linkes Kind] $A[2 \cdot i]$
\item[rechtes Kind] $A[2 \cdot i + 1]$
\end{description}
\end{block}


\end{frame}


\begin{frame}
\frametitle{Rot-Schwarz B"aume}
\begin{alertblock}{Eigenschaften}
\begin{itemize}
\item jeder Knoten hat genau eine Farbe (rot oder schwarz)
\item Wurzel und Bl"atter (NIL) sind schwarz
\item Knoten rot $\Rightarrow$ Kinder schwarz
\item Anzahl schwarzer Knoten im Pfad zu jedem Blatt gleich
\end{itemize}
$\Longrightarrow$ nur wenig unbalanciert
\end{alertblock}

\begin{block}{Rotation}
\begin{itemize}
\item Linksrotation
\item Rechtsrotation
\end{itemize}
\end{block}

\end{frame}


\begin{frame}
\frametitle{Wiederholung}
\begin{enumerate}
\item Bubblesort
\item Selectionsort
\item Insertionsort
\item Mergesort
\item Quicksort
\item Radixsort
\item Heapsort
\item Lineare Suche
\item Bin"are Suche
\item Einfach verkettete Liste
\item Doppelt verkettete Liste
\item Stack
\item Queue
\item Deque
\item Master-Theorem
\end{enumerate}
\end{frame}


\begin{frame}
\frametitle{Ende}
\begin{center}
\textbf{\Huge Vielen Dank f"ur die Aufmerksamkeit!}
\end{center}
\end{frame}


\end{document}

