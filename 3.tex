\documentclass{beamer}
\usepackage{ngerman}
\usepackage{graphicx}
\usepackage{listings}
\usepackage{amsmath}
\usepackage{amssymb}
\usetheme{Madrid}
\title{Algorithmen I Tutorium}
\author{Florian Tobias Schandinat}
\date{05.05.2011}
\institute{FTS}
\lstset{basicstyle=\small\ttfamily,tabsize=4,showstringspaces=false}


\begin{document}


\begin{frame}
\frametitle{Willkommen}
\begin{block}{Algorithmen I Tutorium 19}
\begin{description}
\item[Wer?] Florian Tobias Schandinat\\
\item[Wo?] 50.34, Raum -118\\
\item[Wann?] jeden Donnerstag 15:45-17:15
\end{description}
\end{block}

\begin{block}{Material online}
http://github.com/schandinat/algorithmen1\_ss11
\end{block}
\end{frame}


\begin{frame}
\frametitle{1. "Ubungsblatt}
\begin{block}{$O(1) \not\subseteq O(0)$}
\pause
Annahme: $\exists c \in \mathbb{R_+} \exists n_0 \in \mathbb{N}: 1 \leq c \cdot 0$ \hspace{0.5cm} $\forall n > n_0$\pause\\
da $\alert{1 \not= 0} = c \cdot 0$\hspace{0.5cm} $\forall c \in \mathbb{R} \forall n \in \mathbb{N}$\\
$\Longrightarrow O(1) \not\subseteq O(0)$\\[1cm]
\pause
Anschaulich:
\begin{description}
\item[$O(0)$] der Codeabschnitt wird \alert{nicht} ausgef"uhrt
\item[$O(1)$] der Codeabschnitt wird $k$-Mal ausgef"uhrt, $k \in \mathbb{N}_+$
\end{description}
\end{block}
\end{frame}


\begin{frame}
\frametitle{Erinnerung: Mastertheorem}
\begin{block}{Mastertheorem}
$$T(n) = 1, n = 1$$
$$T(n) = a \cdot T\left(\frac{n}{b}\right) + f(n), n > 1$$
\begin{eqnarray*}
\theta(n^{log_b a})&,& f(n) = O(n^{log_b(a)-\epsilon}), \epsilon > 0\\
\theta(n^{log_b a} \cdot log(n))&,& f(n) = \theta(n^{log_b a})\\
\theta(f(n))&,& f(n) = \Omega(n^{log_b (a)+\epsilon}), \epsilon > 0
\end{eqnarray*}
\end{block}
\end{frame}


\begin{frame}
\frametitle{Mastertheorem -- "Ubung 1}
\begin{exampleblock}{binary search}
PROCEDURE BinaereSuche(A, v, l, r)\\
if $l > r$ then\\
\hspace{1cm}return NIL\\
$m = \lfloor\frac{l + r}{2}\rfloor$\\
if $v == A[m]$ then\\
\hspace{1cm}return m\\
else if $v > A[m]$ then\\
\hspace{1cm}return BinaereSuche(A, v, m + 1, r)\\
else\\
\hspace{1cm}return BinaereSuche(A, v, l, m - 1)\\
end if
\end{exampleblock}

\pause

\begin{block}{}
$T(n) = 1 \cdot T\left(\frac{n}{2}\right) + 1$\pause\\
$\Longrightarrow \theta(log(n))$
\end{block}
\end{frame}


\begin{frame}
\frametitle{Mastertheorem -- "Ubung 2}
\begin{exampleblock}{binary tree traversal}
PROCEDURE tree(n)\\
if $n.left \not= NIL$\\
\hspace{1cm}tree(n.left)\\
if $n.right \not= NIL$\\
\hspace{1cm}tree(n.right)
\end{exampleblock}

\pause

\begin{block}{}
$T(n) = 2 \cdot T\left(\frac{n}{2}\right) + 1$\pause\\
$\Longrightarrow \theta(n)$
\end{block}
\end{frame}


\begin{frame}
\frametitle{Mastertheorem -- Weitere "Ubungen}
\begin{block}{}
$T(n) = 2 \cdot T\left(\frac{n}{2}\right) + n$\pause\\
$\Longrightarrow \theta(n \cdot log(n))$
\end{block}

\pause

\begin{block}{}
$T(n) = 4 \cdot T\left(\frac{n}{2}\right) + n^3$\pause\\
$\Longrightarrow \theta(n^3)$
\end{block}
\end{frame}


\begin{frame}
\frametitle{Amortisierte Analyse}
\end{frame}


\begin{frame}
\frametitle{Hashing -- Grundlagen}
\begin{block}{}
\begin{description}
\item[U] Universalmenge
\item[K] Schl"usselmenge
\item[h] Hashfunktion\\$h : U \rightarrow K$
\end{description}
\pause
\textbf{Hinweis:} Bei $\left|K\right| < \left|U\right|$ \alert{immer} Kollisionen m"oglich 
\end{block}
\end{frame}


\begin{frame}
\frametitle{Hashing -- Kollisionsaufl"osung}
\begin{block}{}
\begin{itemize}
\item Verkettung
\item Offene Adressierung
\begin{itemize}
\item linear
\item quadratisch
\item doppeltes Hashing
\end{itemize}
\end{itemize}
\end{block}
\end{frame}


\begin{frame}
\frametitle{Hashing -- "Ubung}
\begin{exampleblock}{$U = \{ i \in \mathbb{N}_0 | i \leq 20 \}, K = \{ i \in \mathbb{N}_0 | i \leq 6 \}, h(k) = k \mod 7$}
Skizziere die folgenden Operationen mit Verkettung; offener Adressierung mit linearer Konfliktaufl"osung
\begin{itemize}
\item INSERT(2)
\item INSERT(3)
\item INSERT(4)\pause
\item INSERT(10)\pause
\item INSERT(11)\pause
\item DELETE(4)\pause
\item SEARCH(2)\pause
\item SEARCH(5)\pause
\item SEARCH(11)
\end{itemize}
\end{exampleblock}
\end{frame}


\begin{frame}
\frametitle{Ende}
\begin{center}
\textbf{\Huge Vielen Dank f"ur die Aufmerksamkeit!}
\end{center}
\end{frame}


\end{document}

