\documentclass{beamer}
\usepackage{ngerman}
\usepackage{graphicx}
\usepackage{listings}
\usepackage{amsmath}
\usepackage{amssymb}
\usetheme{Madrid}
\title{Algorithmen I Tutorium}
\author{Florian Tobias Schandinat}
\date{07.07.2011}
\institute{FTS}
\lstset{basicstyle=\small\ttfamily,tabsize=4,showstringspaces=false}


\begin{document}


\begin{frame}
\frametitle{Willkommen}
\begin{block}{Algorithmen I Tutorium 19}
\begin{description}
\item[Wer?] Florian Tobias Schandinat\\
\item[Wo?] 50.34, Raum -118\\
\item[Wann?] jeden Donnerstag 15:45-17:15
\end{description}
\end{block}

\begin{block}{Material online}
http://github.com/schandinat/algorithmen1\_ss11
\end{block}
\end{frame}




\begin{frame}
\frametitle{Dynamische Programmierung}
\begin{alertblock}{Grundkonzept}
Zwischenergebnisse speichern und wiederverwenden
\end{alertblock}

\begin{exampleblock}{CYK-Algorithmus}
Ist das Wort $w = bbabaa$ aus $S$ mittels der folgenden Regeln ableitbar
\begin{itemize}
\item $S \rightarrow AB | BC$
\item $A \rightarrow BA | a$
\item $B \rightarrow CC | b$
\item $C \rightarrow AB | a$
\end{itemize}
\end{exampleblock}
\end{frame}


\begin{frame}
\frametitle{Dynamische Programmierung}
\begin{exampleblock}{CYK-Algorithmus}
Ist das Wort $w = bbabaa$ aus $S$ mittels der folgenden Regeln ableitbar
\begin{itemize}
\item $S \rightarrow AB | BC$
\item $A \rightarrow BA | a$
\item $B \rightarrow CC | b$
\item $C \rightarrow AB | a$
\end{itemize}

\begin{tabular}{l|c|c|c|c|c|c}
b & \{B\} & \{\} & \{A\} & \{S,C\} & \{B\} & \{A,\alert{S}\}\\
b & \{B\} & \{S,A\} & \{S,C\} & \{B\} & \{A,S\}\\
a & \{A,C\} & \{S,C\} & \{B\} & \{S,A\}\\
b & \{B\} & \{A,S\} & \{\}\\
a & \{A,C\} & \{B\}\\
a & \{A,C\}
\end{tabular}\\
$\Longrightarrow$ ja\\
{\scriptsize http://de.wikipedia.org/wiki/Cocke-Younger-Kasami-Algorithmus}
\end{exampleblock}
\end{frame}


\begin{frame}
\frametitle{Greedy-Algorithmen}
\begin{block}{Eigenschaften}
\begin{itemize}
\item liefern f"ur viele Probleme keine optimale L"osung
\end{itemize}
\end{block}


\begin{alertblock}{Grundkonzept}
Zu jedem Zeitpunkt die Entscheidung treffen, die dann am Optimalsten erscheint
\end{alertblock}


\begin{exampleblock}{Beispiele}
\begin{itemize}
\item Dijkstra-Algorithmus
\item Algorithmus von Kruskal
\item Heuristik zur L"osung des TSP (nearest neighbour algorithm)
\end{itemize}
\end{exampleblock}
\end{frame}


\begin{frame}
\frametitle{Wiederholung}
\begin{center}
\textbf{\Huge "Ubungen}
\end{center}
\end{frame}


\begin{frame}
\frametitle{Sortieren}
\begin{exampleblock}{}
Sortieren Sie `hans', `peter', `fritz', `horst' und `anna' alphabetisch aufsteigend mittels Radix-Sort, wobei die Namen jeweils mittels ` ' auf die gleiche L"ange aufgef"ullt werden und ` ' kleiner als alle Buchstaben sein soll.

\pause

\begin{itemize}
\item Was unterscheidet Radix-Sort von den anderen Sortierverfahren, die Sie kennen?
\end{itemize}
\end{exampleblock}
\end{frame}


\begin{frame}
\frametitle{Heaps}
\begin{exampleblock}{}
Sortieren Sie 42, 13, 7, 21, 46, 35, 2 mittels Heapsort so dass das Ergebnis eine aufsteigende Reihenfolge hat.

\pause

\begin{itemize}
\item Was bedeutet inplace?
\end{itemize}
\end{exampleblock}
\end{frame}


\begin{frame}
\frametitle{Hashing}
\begin{exampleblock}{}
Es ist die Hashfunktion
$$h(x) = x\ mod\ 7$$
gegeben. Verwenden Sie offene Adressierung mit linearem Sondieren zur Konfliktaufl"osung.
\begin{itemize}
\item F"ugen Sie $0, 7, 1, 14, 6$ ein.
\item Entfernen Sie die $1$.
\end{itemize}
\end{exampleblock}
\end{frame}


\begin{frame}
\frametitle{Ende}
\begin{center}
\textbf{\Huge Vielen Dank f"ur die Aufmerksamkeit!}
\end{center}
\end{frame}


\end{document}

