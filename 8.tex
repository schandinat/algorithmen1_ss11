\documentclass{beamer}
\usepackage{ngerman}
\usepackage{graphicx}
\usepackage{listings}
\usepackage{amsmath}
\usepackage{amssymb}
\usetheme{Madrid}
\title{Algorithmen I Tutorium}
\author{Florian Tobias Schandinat}
\date{16.06.2011}
\institute{FTS}
\lstset{basicstyle=\small\ttfamily,tabsize=4,showstringspaces=false}


\begin{document}


\begin{frame}
\frametitle{Willkommen}
\begin{block}{Algorithmen I Tutorium 19}
\begin{description}
\item[Wer?] Florian Tobias Schandinat\\
\item[Wo?] 50.34, Raum -118\\
\item[Wann?] jeden Donnerstag 15:45-17:15
\end{description}
\end{block}

\begin{block}{Material online}
http://github.com/schandinat/algorithmen1\_ss11
\end{block}
\end{frame}


\begin{frame}
\frametitle{K"urzeste Pfade}

\begin{block}{Ausgangssituation}
\begin{itemize}
\item Graph mit Kantengewichten
\item Startpunkt s
\item \textbf{Gesucht:} k"urzester Pfad von s zu jedem Knoten\\(Pfad mit kleinster Summe an Kantengewichten)
\end{itemize}
\end{block}
\end{frame}


\begin{frame}
\frametitle{K"urzeste Pfade -- Hilfsmittel}

\begin{block}{Relaxation}
\textbf{Gegeben:} Kante (u,v)\\
\textbf{Frage:} Ist der Pfad nach v "uber u k"urzer als der bisher gefundene?\\
$\longrightarrow$ k"urzeren Pfad "ubernehmen
\end{block}
\end{frame}


\begin{frame}
\frametitle{Bellman-Ford-Algorithmus}
\begin{block}{Eigenschaften}
\begin{itemize}
\item erlaubt negative Kantengewichte
\item gibt false bei negativen Zyklen zur"uck
\item Laufzeit: O($|V| \cdot |E|$)
\end{itemize}
\end{block}

\begin{alertblock}{Grundkonzept}
Relaxiere die Kanten $|V| - 1$-Mal in einer beliebigen, festen Reihenfolge
\end{alertblock}
\end{frame}


\begin{frame}
\frametitle{DAG-Shortest-Path}
\begin{block}{Eigenschaften}
\begin{itemize}
\item erlaubt negative Kantengewichte
\item verbietet Zyklen
\item Laufzeit: O($|V| + |E|$)
\end{itemize}
\end{block}

\begin{alertblock}{Grundkonzept}
In topologischer Reihenfolge der Knoten ausgehende Kanten relaxieren
\end{alertblock}
\end{frame}


\begin{frame}
\frametitle{Dijkstra-Algorithmus}
\begin{block}{Eigenschaften}
\begin{itemize}
\item keine negativen Kantengewichte
\item Laufzeit: O($|V|lg|V| + |E|$)
\end{itemize}
\end{block}

\begin{alertblock}{Grundkonzept}
Betrachte immer noch nicht betrachteten Knoten mit k"urzestem Pfad
\end{alertblock}
\end{frame}

\begin{frame}
\frametitle{Ende}
\begin{center}
\textbf{\Huge Vielen Dank f"ur die Aufmerksamkeit!}
\end{center}
\end{frame}


\end{document}

