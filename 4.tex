\documentclass{beamer}
\usepackage{ngerman}
\usepackage{graphicx}
\usepackage{listings}
\usepackage{amsmath}
\usepackage{amssymb}
\usetheme{Madrid}
\title{Algorithmen I Tutorium}
\author{Florian Tobias Schandinat}
\date{12.05.2011}
\institute{FTS}
\lstset{basicstyle=\small\ttfamily,tabsize=4,showstringspaces=false}


\begin{document}


\begin{frame}
\frametitle{Willkommen}
\begin{block}{Algorithmen I Tutorium 19}
\begin{description}
\item[Wer?] Florian Tobias Schandinat\\
\item[Wo?] 50.34, Raum -118\\
\item[Wann?] jeden Donnerstag 15:45-17:15
\end{description}
\end{block}

\begin{block}{Material online}
http://github.com/schandinat/algorithmen1\_ss11
\end{block}
\end{frame}


\begin{frame}
\frametitle{R"uckblick: Hashing}
\begin{alertblock}{Grundprinzip}
Abbildung einer (gro"sen) Menge auf eine (kleinere) Menge
\end{alertblock}

\pause

\begin{exampleblock}{Beispiel 1: Nicht voll-assoziative Caches}
unterer Adressteil
\end{exampleblock}

\pause

\begin{exampleblock}{Beispiel 2: GIT}
SHA1\\
MB $\longrightarrow$ Bytes
\end{exampleblock}

\pause

\begin{block}{Zusatz}
Rehashing
\end{block}
\end{frame}


\begin{frame}
\frametitle{Abschluss: Hashing}
\begin{center}
\textbf{\Huge Fragen?}
\end{center}
\end{frame}


\begin{frame}
\frametitle{Bin"arer Baum}

\begin{block}{Darstellung im Array A[i]}
Beziehungen \textbf{(sofern vorhanden)}
\pause
\begin{description}
\item[Elternknoten] $A[\lfloor\frac{i}{2}\rfloor]$
\item[linkes Kind] $A[2 \cdot i]$
\item[rechtes Kind] $A[2 \cdot i + 1]$
\end{description}
\end{block}

\pause

\begin{exampleblock}{Wie sieht der zugeh"orige Baum aus?}
\begin{tabular}{ | c | c | c | c | c | c | }
\hline
5 & 2 & 7 & 3 & 8 & 4\\
\hline
\end{tabular}
\end{exampleblock}
\end{frame}


\begin{frame}
\frametitle{Heaps}
\begin{block}{Eigenschaft}
Schneller Zugriff auf das kleinste/gr"o"ste Element
\end{block}

\pause

\begin{alertblock}{Ist ein absteigend/aufsteigend sortiertes Array ein Maximum/Minimum-Heap?}
\pause
Ja!
\end{alertblock}

\pause

\begin{exampleblock}{"Ubung: Maximum-Heap}
\begin{itemize}
\item Bauen Sie einen Maximum-Heap aus
\begin{tabular}{ | c | c | c | c | c | c | }
\hline
5 & 2 & 7 & 3 & 8 & 4\\
\hline
\end{tabular}

\pause

\item F"ugen Sie der Reihe nach folgenden Elemnte ein: 1, 9

\pause

\item Entfernen Sie das Maximum
\end{itemize}
\end{exampleblock}
\end{frame}


\begin{frame}
\frametitle{Heaps -- Zusammenfassung}
\begin{block}{Heap Operationen}
\begin{description}[EXTRACT-MAXIMUM]
\item[BUILD-MAX-HEAP] $O(n)$
\item[INSERT] $O(log(n))$
\item[MAXIMUM] $O(1)$
\item[EXTRACT-MAXIMUM] $O(log(n))$
\end{description}
\end{block}
\end{frame}


\begin{frame}
\frametitle{Ende}
\begin{center}
\textbf{\Huge Vielen Dank f"ur die Aufmerksamkeit!}
\end{center}
\end{frame}


\end{document}

