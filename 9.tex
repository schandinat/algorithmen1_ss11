\documentclass{beamer}
\usepackage{ngerman}
\usepackage{graphicx}
\usepackage{listings}
\usepackage{amsmath}
\usepackage{amssymb}
\usetheme{Madrid}
\title{Algorithmen I Tutorium}
\author{Florian Tobias Schandinat}
\date{30.06.2011}
\institute{FTS}
\lstset{basicstyle=\small\ttfamily,tabsize=4,showstringspaces=false}


\begin{document}


\begin{frame}
\frametitle{Willkommen}
\begin{block}{Algorithmen I Tutorium 19}
\begin{description}
\item[Wer?] Florian Tobias Schandinat\\
\item[Wo?] 50.34, Raum -118\\
\item[Wann?] jeden Donnerstag 15:45-17:15
\end{description}
\end{block}

\begin{block}{Material online}
http://github.com/schandinat/algorithmen1\_ss11
\end{block}
\end{frame}


\begin{frame}
\frametitle{Disjunkte Mengen}

\begin{block}{Operationen}
\begin{itemize}
\item MAKE-SET -- Menge anlegen
\item FIND-SET -- Repr"asentant von Menge abfragen
\item UNION -- Mengen vereinigen
\end{itemize}
\end{block}
\end{frame}


\begin{frame}
\frametitle{Minimale Spannb"aume (MST)}

\begin{block}{Ausgangssituation}
\begin{itemize}
\item ungerichteter zusammenh"angender Graph
\item Kantengewichte
\item \textbf{Gesucht:} Kanten, die alle Knoten verbinden, so dass Gesamtgewicht davon minimal
\end{itemize}
\end{block}

\pause

\begin{alertblock}{Algorithmus von Kruskal}
F"uge Kante mit geringstem Gewicht hinzu, die zwei Partitionen verbindet
\end{alertblock}


\begin{alertblock}{Algorithmus von Prim}
F"uge Kante mit geringstem Gewicht hinzu, die aus der Menge der erreichbaren Knoten hinausf"uhrt
\end{alertblock}


\begin{block}{}
Laufzeit: O(E lg V)
\end{block}
\end{frame}


\begin{frame}
\frametitle{Dynamische Programmierung}
\begin{alertblock}{Grundkonzept}
Zwischenergebnisse speichern und wiederverwenden
\end{alertblock}
\end{frame}


\begin{frame}
\frametitle{Greedy-Algorithmen}
\begin{block}{Eigenschaften}
\begin{itemize}
\item liefern f"ur viele Probleme keine optimale L"osung
\end{itemize}
\end{block}


\begin{alertblock}{Grundkonzept}
Zu jedem Zeitpunkt die Entscheidung treffen, die dann am Optimalsten erscheint
\end{alertblock}
\end{frame}


\begin{frame}
\frametitle{Ende}
\begin{center}
\textbf{\Huge Vielen Dank f"ur die Aufmerksamkeit!}
\end{center}
\end{frame}


\end{document}

